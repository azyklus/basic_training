\section{Programming Languages}
\paragraph{}
   Although this report will focus mostly on the C programming language, we will discuss and use concepts from other languages like Python, Rust, C\#,
   and others.

\paragraph{}
   The C programming language is widely considered to be the father of modern programming languages and programming language theory. Learning C is
   probably the best way to become a successful programmer and can massively benefit you in the long term as learning C introduces you to concepts
   that are heavily utilized in the world of computer science.

\paragraph{}
   Languages like Python, however, offer much of the same in that regard. Python is heavily used in data science and mathematics as the goto language
   for said fields because of its concise syntax and gargantuan ecosystem where modules and packages are concerned.

\paragraph{}
   Rust is a very new language, relatively speaking. Having been created in 2010 by Graydon Hoare at Mozilla, it solves quite a few problems that've
   emerged through rigorous usage of languages like C over the years. One such problem would be memory management. In C, mistakes in memory management
   can often be made without ever even realizing what you've done. Rust solves this by implementing a set of syntax rules like borrow-checking, which
   frees unused memory the instant it goes out of scope. The methods that the Rust language employs to rid programs of memory-related bugs and
   problems are novel, and time will tell if they hold up to their intended purpose.

\section{What You Need}
   \subsection{Compiler}
   \paragraph{}
      When getting started with this endeavor, you'll need a few things to ensure that you can follow alongside our work. One thing you'll need is a C
      compiler; C compilers are rather ubiquitous and often come preinstalled on most operating systems with Windows being a glaring exception. If you
      are running on a Windows system, then you'll be able to find a C compiler \href{https://github.com/llvm/llvm-project/releases/download/llvmorg-12.0.1/LLVM-12.0.1-win64.exe}{here}.

   \subsection{Text Editor}
   \paragraph{}
      You will also need a text editor. \href{https://code.visualstudio.com}{Visual Studio Code} is a great free editor that you can use and hit the
      ground running. Some more options include \href{https://sublimetext.com}{Sublime Text} and \href{https://atom.io}{Atom}. Each of these editors have
      available extensions that allow you to customize your editing experience to your desired degree.

   \subsection{Command Line Interface / Interpreter}
   \paragraph{}
      This one is very much based on the opinion of the individual, but I personally prefer \href{https://docs.microsoft.com/en-us/powershell}{PowerShell} for most things and it really is the only option
      available on Windows unless you don't mind using CMD.EXE or the \href{https://docs.microsoft.com/en-us/windows/wsl/about}{Windows Subsytem for Linux}.
      If you are running on a Linux or Mac system, you have considerably more options in the form of \textit{bash}, \textit{zsh}, \textit{fish}, and others.
      Feel free to explore and find ones that meet your personal requirements.

\section{Continuing}
\paragraph{}
   Moving forward, most of the contents of this document will involve simple demonstrations and sometimes interactive lessons in programming with C, Python,
   and potentially some Bash/PowerShell scripting. Having your text editor and your CLI open will be most helpful if you intend to follow along with each
   figure in the document.
