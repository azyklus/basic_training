\section{What You Need}
   \subsection{Compiler}
   \paragraph{}
      When getting started with this endeavor, you'll need a few things to ensure that you can follow alongside our work. One thing you'll need is a C
      compiler; C compilers are rather ubiquitous and often come preinstalled on most operating systems with Windows being a glaring exception. If you
      are running on a Windows system, then you'll be able to find a C compiler \href{https://github.com/llvm/llvm-project/releases/download/llvmorg-12.0.1/LLVM-12.0.1-win64.exe}{here}.

   \subsection{Text Editor}
   \paragraph{}
      You will also need a text editor. \href{https://code.visualstudio.com}{Visual Studio Code} is a great free editor that you can use and hit the
      ground running. Some more options include \href{https://sublimetext.com}{Sublime Text} and \href{https://atom.io}{Atom}. Each of these editors have
      available extensions that allow you to customize your editing experience to your desired degree.

   \subsection{Command Line Interface / Interpreter}
   \paragraph{}
      This one is very much based on the opinion of the individual, but I personally prefer \href{https://docs.microsoft.com/en-us/powershell}{PowerShell} for most things and it really is the only option
      available on Windows unless you don't mind using CMD.EXE or the \href{https://docs.microsoft.com/en-us/windows/wsl/about}{Windows Subsytem for Linux}.
      If you are running on a Linux or Mac system, you have considerably more options in the form of \textit{bash}, \textit{zsh}, \textit{fish}, and others.
      Feel free to explore and find ones that meet your personal requirements.
