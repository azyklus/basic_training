\section{Data Types}

\paragraph{}
   In this section, we are going to talk about data types.

\paragraph{}
   \begin{displayquote}
      Data types in C refer to an extensive system used for declaring variables and functions of different ``types''. The type of a variable or
      function determines how much space it occupies in storage and how the bit pattern stored is interpreted.
   \end{displayquote}

\paragraph{}
   The types in C can be classified as follows:
   \begin{center}
      \begin{tabular}{ ||p{2in}| p{4in}|| }
         \hline
         No.Types & description\\ [1.5ex]
         \hline\hline
         1.) Basic types & They are basic arithmetic types and can be further classified into (a) integer types and (b) floating-point types;\\
         \hline
         2.) Enumerated types & They are, again, arithmetic types and they are used to define variables that can only assign certain discrete integer
         values throughout the program;\\
         \hline
         3.) Void Type & This type indicates the lack of a specified value;\\
         \hline
         4.) Derived Types & These include (a) Pointer types, (b) Array types, (c) Structure types, (d) Union types, and (e) Function types.\\
         \hline
      \end{tabular}
   \end{center}

% There is a \newpage here to keep Subsection 2.2.1 from being cut in half.
\newpage

\rule{\textwidth}{0.3ex}\par

\subsection{Integer Types}
   \paragraph{}
      The following table provides the details of standard integer types with their storage sizes and value ranges −
      \begin{center}
         \begin{tabular}{ ||p{1in}| p{1in} | p{4in}|| }
            \hline
            \textbf{Type} & \textbf{Storage Size} & \textbf{Value range} \\ [1.75ex]
            \hline\hline
            char & 1 byte & -128 to 127 or 0 to 255 \\
            \hline
            unsigned char & 1 byte & 0 to 255 \\
            \hline
            signed char & 1 byte & -128 to 127 \\
            \hline
            int & 2 or 4 bytes & -32,768 to 32,767 or -2,147,483,648 to 2,147,483,647 \\
            \hline
            unsigned int & 2 or 4 bytes & 0 to 65,535 or 0 to 4,294,967,295 \\
            \hline
            short & 2 bytes & -32,768 to 32,767 \\
            \hline
            unsigned short & 2 bytes & 0 to 65,535 \\
            \hline
            long & 4 or 8 bytes & -9223372036854775808 to 9223372036854775807 \\ [0.8ex]
            \hline
            unsigned long & 8 bytes & 0 to 18446744073709551615 \\
            \hline
         \end{tabular}
      \end{center}

\paragraph{}
   To get the precise size of a type or variable on a particular platform, you can use the \textbf{sizeof} operator.
   The expressions \textit{sizeof(<type>)} yields the storage size of the object or type in bytes. On the next page, I will provide an example of 
   this operator in action.

% There is a \newpage here to prevent the figure from being cut in half.
\newpage

\begin{lstlisting}
// This is an include directive.
// It tells the compiler to 'include' the contents of a file in 
// the standard library called "stdio.h" into our program.
#include <stdio.h>

// Hope you recognize this.
// I spoke in Section 2.1 "Hello World" about the 'main' 
// function and its purpose. Feel free to revisit the 
// section if you need a refresher on the purpose
// of this specially named function.
int main(int argc, char** argv)
{
   // INT data type.
   int a = 1;
   printf("Size of variable a : %d\n", sizeof(a));
   printf("Size of data type INT : %d\n", sizeof(int));
   // CHAR data type.
   char b = 'b';
   printf("Size of variable b : %d\n", sizeof(b));
   printf("Size of data type CHAR : %d\n", sizeof(char));
   // FLOAT data type.
   float c = 1.0;
   printf("Size of variable c : %d\n", sizeof(c));
   printf("Size of data type FLOAT : %d\n", sizeof(float));
   // DOUBLE data type.
   double d = 1.0;
   printf("Size of variable d : %d\n", sizeof(d));
   printf("Size of data type DOUBLE : %d\n", sizeof(double));
   // SHORT data type.
   short e = 1;
   printf("Size of variable e : %d\n", sizeof(e));
   printf("Size of data type SHORT : %d\n", sizeof(short));

   return 0;
}
\end{lstlisting}

\paragraph{}
   In these examples, I will assume that you are compiling the code as I have and are running the examples alongside of me.

\paragraph{}
   When we execute this program, we should see the following output:

\begin{lstlisting}
Size of variable a : 4
Size of data type INT : 4
Size of variable b : 1
Size of data type CHAR : 1
Size of variable c : 4
Size of data type FLOAT : 4
Size of variable d : 8
Size of data type DOUBLE : 8
Size of variable e : 2
Size of data type SHORT : 2
\end{lstlisting}

\paragraph{}
   It is important to note that the storage sizes that we get from \textbf{sizeof} are represented in \textit{bytes} and not \textit{bits}.
   This means that instead of 32 being the size we see for \textit{INT}, we get \textit{4}. If you would like to make sure for yourself that we are
   seeing the sizes I mentioned earlier, feel free to multiply all of the numbers in the program's output by 8.

\begin{center}
   \begin{tabular}{ || p{1in} | p{2in} | p{2in} || }
   \hline
   \textbf{Type} & \textbf{Out x Bytes} & \textbf{Bits} \\ [0.75ex]
   \hline\hline
   INT & \(4 * 8\) & 32 \\
   \hline
   CHAR & \(1 * 8\) & 8 \\
   \hline
   FLOAT & \(4 * 8\) & 32 \\
   \hline
   DOUBLE & \(8 * 8\) & 64 \\
   \hline
   SHORT & \(2 * 8\) & 16 \\
   \hline
   \end{tabular}
\end{center}

% A \newpage is placed here to avoid cutting the following subsection in half.
\newpage

\subsection{Void Types}
   \paragraph{}
      The \textit{void} type indicates the lack of a specified value. The following table describes where \textit{void} may be used:
      \begin{center}
         \begin{tabular}{ || p{2in} | p{4in} || }
         \hline
         \textbf{Type} & \textbf{Description} \\ [1ex]
         \hline\hline
         Function: Return Argument & There are various functions in C which do not return any value or you can say they return void. A function with
         no return value has the return type as void. For example, \verb;void exit(int status); \\ [0.4ex]
         \hline
         Function: Argument as Void & There are various functions in C which do not accept any parameter. A function with no parameter can accept a
         void. For example, \verb;int rand(void); \\ [0.6ex]
         \hline
         Pointers to Void & A pointer of type void * represents the address of an object, but not its type. For example, a memory allocation function 
         called \textit{malloc}, which returns a pointer to void which can be casted to any data type. \\ [0.4ex]
         \hline
         \end{tabular}
      \end{center}

   \paragraph{}
      The \textit{void} type is often used in conjunction with a pointer, as noted above, to allow the programmer to cast into any type.
      In this next part, we are going to be discussing structure types and how they are used to organize data within a program.

% A \newpage is placed here to separate subsections.
\newpage

\subsection{Structure Types}
   \paragraph{}
      TODO: Complete this segment as time permits.

% A \newpage is placed here to separate subsections.
\newpage

\subsection{Final Thoughts}
   \paragraph{}
      Data types are a relatively simple concept to grasp if you take the time to understand why they are useful.
      It takes time and often requires more patience than you are willing to offer, but data types make programming so much more simple
      and provide a window into understanding what the computer is doing with your program.

% A \newpage should be placed at the end of every section.
\newpage
